\chapter{REVIEW OF RELATED LITERATURE}
This chapter presents relevant information and related studies that support the development of the proposed system.

\section{Cacao Diseases and Diagnosis}
Cacao (\textit{Theobroma cacao}) is highly vulnerable to various diseases that threaten yield and production quality. One of the most aggressive fungal pathogens is \textit{Phytophthora megakarya}, responsible for black pod disease, as discussed by \cite{Andrews1997}. This disease affects all parts of the cacao plant, including pods, leaves, and stems, particularly in humid conditions.

In West Africa, \textit{P. megakarya} is a major threat, whereas in the Philippines, a different variant, \textit{Phytophthora palmivora} (P.\textit{Palmivora}), is the primary cause of pod rot, as mentioned by \cite{Solpot2020}. This pathogen was first documented in Luzon in 1918 by Reinking and remains a significant challenge for local farmers. According to \cite{AceboGuerrero2012}, \textit{P. palmivora} can cause annual losses of 20–30\%, with severe cases reaching up to 90\% under high humidity.

Studies suggest that infected cacao plants can contribute to the spread of the disease to neighboring trees. Field experiments conducted by \cite{Ndoumbe2004} demonstrated that pod removal reduced black pod incidence by 22\% and 31\% in the first year and by 9\% and 11\% in the second year, confirming the role of contaminated pods in disease transmission. However, research by \cite{Babin2018} later revealed that insect pests, particularly \textit{Helopeltis bakeri}, a mirid bug, also facilitate the spread of fungal pathogens. These insects feed on pods and shoots, causing severe damage and creating potential entry points for fungal infections, as supported by \cite{Guest2007}.

Other notable diseases affecting cacao include Cacao Swollen Shoot Virus (CSSV), a viral disease that causes swelling of shoots and veins; Vascular-Streak Dieback (VSD), a fungal infection that results in yellowing and leaf shedding; Witches' Broom Disease (\textit{Moniliophthora perniciosa}), which causes abnormal shoot proliferation ('brooms') and negatively impacts pod development; and Frosty Pod Rot (\textit{Moniliophthora roreri}), which appears as white fungal growth on pods, leading to pod rot and significant crop losses.

Farmers and researchers distinguish between healthy cacao plants and those infected with diseases by observing specific visual symptoms on various parts of the plant, such as leaves, pods, stems, and roots. Initial signs include small, circular brown spots on the pod surface, as described by the \cite{MinistryAgri}). These spots rapidly expand, turning dark brown or black, eventually covering the entire pod. Under wet conditions, white fungal growth may appear on the lesions. Infected pods often emit a characteristic fishy odor and, if untreated, become blackened and mummified.

\section{Current Approaches to Cacao Disease Detection and Quality Control}
\subsection*{Traditional Methods}
The \cite{PhilCacaoRoadmap2021} highlighted that most cacao farms in the country are owned and managed by smallholding farmers, many of whom have acquired farming knowledge through ancestral practices or personal experience. This includes manually distinguishing between cacao pods with black pod rot, then separating them after harvest. However, this approach has limitations. According to \cite{ForestPhytophthora}, once exposed to pathogens, healthy pods may develop internal infections within 15 days, making early detection and intervention crucial. Delays in identifying and removing infected pods reduce the effectiveness of disease management efforts and increase the likelihood of disease spread, especially in pre-harvest settings.

To mitigate cacao diseases, farmers employ several cultural and chemical control methods including sanitation and pruning, according to \cite{AceboGuerrero2012}. Pruning improves air circulation and reduces humidity, creating less favorable conditions for fungal growth. On the other hand, sanitation involves removing diseased pods and plant debris, helping eliminate sources of inoculum (increase immunity to a disease) and prevents reinfection. Additionally, \cite{Merga2022} highlighted that farmers frequently harvest cacao pods to reduce the inoculum load of pathogens such as \textit{Phytophthora spp.}, thereby minimizing disease transmission. Scientists and fungicide experts developed copper-based compounds and metalaxyl to control black pod disease, as reported by \cite{Aneani2007}. When combined with crop sanitation, fungicide application has been shown to significantly reduce disease incidence and improve yields.

\subsection*{Destructive and Non-destructive Disease Detection}
In cacao classification, traditional destructive techniques are often employed to assess the quality of cocoa beans. Among these methods, \cite{Nguyen2022} reported that the cut-test stands out as the most widely used due to its simplicity and effectiveness—it involves slicing beans to inspect mold, germination, and fermentation. Though practical, it is labor-intensive and less precise. With this, \cite{Quelal2020} introduced chromatographic analysis, which offers a higher level of sensitivity, capable of detecting metabolites and contaminants even in trace quantities. While highly reliable, chromatographic analysis is more complex and requires specialized equipment, making it less accessible for routine quality control in cacao production. Since these approaches are destructive in nature, they result in the loss of sampled beans, making it less ideal for large-scale quality control.

In contrast, \cite{Alvarado2023} listed non-destructive techniques such as imaging sensors, spectroscopy, and thermal imaging to monitor plant health. These methods facilitate the early detection of diseases, allowing for timely interventions that can prevent the spread of infections and reduce crop damage.

For instance, a study on ginseng root diseases highlighted the urgency of developing efficient non-destructive testing methods for early-stage detection and limiting further spread. \cite{Silva2024} explored edge computing for real-time classification of leaf diseases using thermal imaging, which converts infrared (IR) radiation into visible images. Near-Infrared (NIR) Spectroscopy and Hyperspectral Imaging (HSI) also analyze internal attributes of cacao, such as moisture content, fermentation levels, and fat composition \cite{Alvarado2023}. Additionally, imaging-based computer vision detects surface defects and size variations, enhancing quality control, reducing waste, and ensuring better cacao processing outcomes.

\subsection*{Pre-harvest Disease Detection}
Pre-harvest disease detection is significant in maintaining crop health and preventing losses. Traditional methods, such as manual inspection, have been shown to be labor-intensive, time-consuming, and prone to human error, especially in large-scale cacao farming. According to \cite{Tan2018}, these limitations often result in delayed detection and intervention, which can lead to the spread of diseases like P.\textit{Palmivora}, a major cause of black pod disease in cacao. Researchers have emphasized the importance of early detection tools to address these challenges, highlighting that timely interventions can significantly reduce crop damage and yield loss. According to \cite{Yadav2024}, UAVs equipped with high-resolution cameras and multispectral sensors offer an efficient solution, enabling farmers to monitor large areas in real-time and detect disease symptoms before they become widespread. This technology allows for early identification of subtle symptoms, such as discoloration or texture changes in leaves and pods, which are often missed by manual inspections, as pointed out by \cite{Upadhyay2025}.

The integration of UAVs in precision agriculture has been increasingly recognized for its ability to enhance the accuracy and speed of disease detection. According to \cite{Vyas2023}, UAVs can cover vast agricultural areas quickly, providing farmers with comprehensive, up-to-date data. This system not only improves the timeliness of disease detection but also enables targeted actions to mitigate disease outbreaks. According to \cite{Taesiri2023}, by combining UAV technology with advanced image processing and geotagging, the detection process becomes more precise, ensuring that even subtle and atypical symptoms are identified. This proactive approach to disease management during the pre-harvest phase is essential for reducing crop losses and improving the overall sustainability and productivity of cacao farming.

\subsection*{Computer-aided Cacao Disease Detection Technology in Agriculture}
Recent advancements in cacao plant disease detection have leveraged artificial intelligence (AI) and image processing to improve classification accuracy and disease management. These methods, as stated by \cite{Upadhyay2025}, automatically learn and extract intricate features from raw image data, capturing subtle patterns associated with specific diseases. They surpass traditional manual methods, leading to improved detection accuracy.

One such study by \cite{BaculioBarbosa2022} introduced an objective classification system for cacao pods. Their model combines Local Binary Pattern (LBP) features and a Color Histogram (CH) with an Artificial Neural Network (ANN) to differentiate between healthy and unhealthy pods. This research underscores the importance of machine learning in agricultural diagnostics, ensuring that disease identification is both efficient and accurate. Similarly, \cite{Tan2018} developed AuToDiDAC, an automated mobile tool designed to detect black pod rot, one of the most devastating diseases affecting cacao plants.

Another notable contribution comes from \cite{Basri2020}, who proposed a mobile image processing application for identifying pests and diseases in cacao fruit. Their deep learning-powered system processes real-time images, categorizing fruits as healthy, pest-infected, or disease-affected with high accuracy. A subsequent review by \cite{Basri2020} further explored the role of image processing in modern cocoa plantations, revealing that color-based disease detection models achieve an average accuracy of 82.85\%. The authors advocate for integrating machine learning models to optimize disease prediction and support precision agriculture.

Meanwhile, \cite{Buenano2024} proposed a deep learning model for diagnosing monilia and black pod diseases in cacao pods. Using EfficientDet-Lite4, their model was trained on a dataset of diseased and healthy pods and later deployed in a mobile application to assist farmers. The app's intuitive design and real-time disease identification make it a valuable tool, particularly for small-scale farmers with limited access to expert guidance.

Building on these innovations, \cite{Tovurawa2025} applied deep learning and Exploratory Data Analysis (EDA) to cacao disease detection. Their research focused on a custom Convolutional Neural Network (CNN) that outperformed other models in accurately classifying cacao diseases. By utilizing a dataset from Ghanaian cacao farms, this study provides a locally relevant solution that could enhance crop resilience and farmer livelihoods.

\section{Unmanned Aerial View (UAV) Technology in Agriculture}
The advancement of drone-based technologies and deep learning algorithms has significantly contributed to precision agriculture, particularly in large-scale crop monitoring and quality assessment. \cite{Alam2022} developed a drone-based crop product quality monitoring system that utilizes vision cameras and a Gaussian kernel support vector machine (SVM) to classify vegetables into rotten and non-rotten categories. Their approach extracts chromatic, contour, and texture features from image datasets of tomatoes, cauliflower, and eggplants to enhance classification accuracy. The system demonstrated a 97.9\% true positive rate and 95.4\% overall accuracy, highlighting its effectiveness in agricultural product quality monitoring.

Similarly, \cite{Mazzia2020} explored the refinement of satellite-driven vegetation indices using UAVs and machine learning. While satellite imagery plays a crucial role in crop monitoring, its limitations in capturing detailed intra-row crop variations can lead to inaccuracies. Their study proposed a deep learning framework that integrates high-resolution UAV multispectral data to improve Normalized Difference Vegetation Index (NDVI) maps. By training a convolutional neural network (CNN) with UAV-acquired datasets, their approach produced refined NDVI maps that provided more accurate crop status assessments. This refinement enabled the generation of 3-class vineyard vigor maps using a K-means classifier, offering a valuable tool for site-specific crop management.

In a related study, \cite{Vardhan2023} introduced a deep learning-based approach for detecting plant diseases using drone-captured imagery. Their model utilized a convolutional neural network (CNN) trained on a comprehensive dataset of plant species exhibiting various diseases. The study demonstrated high proficiency in disease classification and detection, even under challenging imaging conditions. By integrating real-time drone monitoring with deep learning, their approach presents a scalable and efficient solution for improving plant health assessment, further advancing the role of smart agriculture in modern farming practices.

\section{You Only Look Once versions (YOLO) for Object Detection}
The You Only Look Once (YOLO) framework has become one of the most influential deep learning models for real-time object detection. Its successive versions have continuously evolved to address limitations in accuracy, speed, and computational efficiency. According to Ultralytics \cite{Terven2023}, the YOLO family progressed from YOLOv1 to the most recent YOLO11, each version integrating new architectural strategies and optimization methods to improve detection across diverse applications.

YOLOv8, released in January 2023, introduced an anchor-free decoupled head design, separating classification from bounding box regression. This design enhanced optimization, particularly for small-object detection, and allowed flexible deployment through multiple model sizes (Nano to X), enabling a tradeoff between speed and accuracy \cite{Wang2023}. YOLOv8 also extended its capabilities beyond detection to include classification and segmentation, making it versatile in agricultural monitoring and disease diagnosis tasks \cite{Vyas2023}.
YOLOv8, released in January 2023, introduced an anchor-free decoupled head design that separates classification from bounding box regression. This architectural refinement enhanced optimization, particularly for small-object detection, while supporting flexible deployment across multiple model sizes (Nano to X), enabling a balance between speed and accuracy, as discussed by \cite{Wang2023}. In addition, \cite{Vyas2023} highlighted that YOLOv8 extended its functionality beyond detection to include classification and segmentation, making it highly versatile for agricultural monitoring and disease diagnosis tasks.

YOLOv9, launched in February 2024, improved feature aggregation by incorporating \textit{Programmable Gradient Information (PGI)} and the \textit{Generalized Efficient Layer Aggregation Network (GELAN)}. These innovations enhanced the retention of gradient information across layers, thereby boosting detection accuracy while maintaining efficiency for edge devices and mobile platforms, according to \cite{UltralyticsBlog2025}. Furthermore, YOLOv10, released in May 2024, further optimized object detection by introducing a \textit{Non-Maximum Suppression (NMS)-free} design through consistent dual-label assignment. This improvement allowed more stable predictions across multiple objects, eliminating redundant bounding boxes and enhancing both training efficiency and precision in dense detection environments, as reported by \cite{UltralyticsBlog2025}.

The latest version, YOLOv11, released in September 2024, emphasized efficiency and scalability. With approximately 22\% fewer parameters than YOLOv8m while achieving higher mean Average Precision (mAP), YOLOv11 demonstrated superior accuracy-to-complexity performance. Benchmark evaluations confirmed its ability to outperform YOLOv8, YOLOv9, and YOLOv10 in both detection accuracy and inference speed, making it well-suited for large-scale and resource-constrained agricultural applications, as described by \cite{UltralyticsBlog2025}.

\begin{longtable}{p{8cm} r}
	\caption{\textit{Comparison of YOLO Versions}} \label{tab:yolo_comparison} \\
	
	\toprule
	\textbf{YOLO Version and Key Features} & \textbf{Release Year} \\
	\midrule
	\endfirsthead
	
	\toprule
	\textbf{YOLO Version and Key Features} & \textbf{Release Year} \\
	\midrule
	\endhead
	
	\bottomrule
	\endfoot
	
	YOLOv8 — Introduced an anchor-free and decoupled head architecture; enhanced small-object detection and supported multi-task learning (detection, classification, segmentation). & 2023 \\
	\midrule
	YOLOv9 — Integrated Programmable Gradient Information (PGI) and GELAN for improved gradient preservation, feature aggregation, and detection accuracy on edge devices. & 2024 (February) \\
	\midrule
	YOLOv10 — Adopted a Non-Maximum Suppression (NMS)-free framework with consistent dual-label assignment; improved stability and precision in dense detection scenarios. & 2024 (May) \\
	\midrule
	YOLOv11 — Achieved 22\% parameter reduction compared to YOLOv8m while attaining higher mean Average Precision (mAP); optimized for scalability, efficiency, and inference speed. & 2024 (September) \\
\end{longtable}

Across the YOLO series, continuous improvements will be made in terms of accuracy and efficiency. Earlier versions such as YOLOv3 and YOLOv4 will establish the foundation for reliable object detection, while YOLOv8 will introduce a more efficient, anchor-free design. According to \cite{UltralyticsBlog2025}, YOLOv11 will offer the best balance of speed, accuracy, and adaptability, making it suitable for detecting small and complex objects. For this study, which will focus on identifying subtle cacao pod lesions under varying environmental conditions, YOLOv11 will be considered the most appropriate version since it will provide faster processing and higher detection precision, both of which are essential for real-time UAV operations.

\section{Geotagging using QGIS in Agriculture}
Geotagging plays a vital role in precision agriculture by connecting crop-related data with specific locations in the field. This spatial connection helps farmers make smarter, more targeted decisions about managing their crops. For example, \cite{Mohidem2021} showed that combining geotagged aerial images with vegetation indices significantly improves how accurately crop conditions can be monitored. This makes it easier to spot and address issues in particular areas of the field. Likewise, \cite{Rahman2021} used QGIS software to geotag individual durian trees in Malaysia’s Kluang Valley. By linking each tree’s location with detailed ground data—such as tree height, canopy size, soil pH, and leaf dimensions—the study created a more precise and efficient way to monitor the health and growth of each tree. This method allowed for a more organized and data-driven approach to plantation monitoring by enabling the spatial joining of field measurements with geographic coordinates. \cite{Rahman2021} highlights the effectiveness of QGIS in supporting precise andefficient geotagging workflows, underscoring its relevance as a valuable tool in agricultural decision-making and crop health assessment.

\section{Synthesis}
Historically, cacao farmers and researchers have relied on traditional disease management methods such as visual inspection, pruning, sanitation, and fungicide application (\cite{AceboGuerrero2012, Aneani2007}). However, these approaches have limitations, including reliance on farmer expertise, risk of misdiagnosis, and environmental concerns from fungicide use.

To improve accuracy, laboratory-based techniques like PCR and chromatographic analysis (\cite{Nguyen2022, Quelal2020}) have been used, but they are often time-consuming, destructive, and impractical for field applications. This has driven interest in non-destructive methods, including remote sensing, imaging, and AI-based approaches.

Recent research has explored hyperspectral, thermal, and fluorescence imaging for early disease detection, leveraging plant stress-induced changes in light absorption and reflection (\cite{Alvarado2023}; \cite{Silva2024}). Meanwhile, AI and deep learning, particularly CNNs, have enhanced automatic disease classification (\cite{BaculioBarbosa2022}). Notable developments include a system integrating LBP and Color Histogram with ANN for cacao pod classification and \cite{Tan2018} AuToDiDAC, a mobile app for black pod disease detection.

Despite these advancements, challenges persist. Many existing studies rely on static images, which may fail to capture the full morphology of cacao pods, leading to incomplete assessments. Moreover, mobile-based approaches, while accurate, often face limitations in convenience and coverage. To overcome these issues, this study proposes the integration of a UAV system with the YOLO object detection framework and CNN for automated, real-time cacao pod disease detection. Based on comparative analysis, YOLOv11 demonstrates the most effective balance between speed, precision, and scalability, particularly in detecting small and complex objects (\cite{UltralyticsBlog2025}). Therefore, YOLOv11 is recommended as the optimal version for this study, offering superior detection accuracy, computational efficiency, and adaptability for UAV-based agricultural applications.