\chapter{INTRODUCTION}

{\baselineskip=2\baselineskip
Dual – mode telemetry presents new ability to the UAV communication creating a new way of handling disconnection and fail safe protocol. This research may solve the occlusion problems with the radio frequency communication and the handshaking problems in the mobile network connection. 

This paper is dedicated to present the concept of the integrated communication system for UAV (Unmanned Aerial Vehicle) data and telemetry transmission. The aim of this paper is to present a working prototype UAV with working dual mode communication that involves the telemetry, FPV (First – Person View), and the RC (Remote Control). In the paper, both the concept of the system and elements of its realization are presented.  \citep{Bhattacharya2004}

According to \citep{Fuller2014} ... 


\section{Background of the Study}

Telemetry, derived from the Greek roots "tele" (remote) and "metron" (measure), is the process of recording and transmitting the readings of instruments and devices from remote or inaccessible points to an IT system in a different location for monitoring and analysis. This technology has become a cornerstone in various fields such as aerospace, healthcare, environmental science, and industrial applications due to its ability to provide real-time data monitoring, diagnostics, and predictive maintenance \citep{Butcher2014}.

The primary function of telemetry systems is to collect data from sensors and transmit it to a centralized system for analysis. This process involves several key components: sensors to capture data, transmitters to send the data, receivers to collect the data, and a central processing unit to analyze and store the data. With the advent of the Internet of Things (IoT) and advancements in wireless communication technologies, telemetry has evolved significantly, enabling more efficient and comprehensive data acquisition and monitoring systems. 

In aerospace, telemetry is crucial for monitoring the status and health of spacecraft and satellites, providing data on parameters such as temperature, pressure, and velocity. In healthcare, telemetry systems are used to monitor patients' vital signs remotely, allowing for timely medical interventions and reducing the need for prolonged hospital stays. Environmental telemetry systems play a pivotal role in tracking weather conditions, pollution levels, and natural disaster warnings, contributing to better disaster management and environmental protection.

The integration of telemetry in industrial applications has revolutionized how industries operate. Through real-time monitoring of machinery and processes, industries can minimize downtime, optimize performance, and enhance safety. Predictive maintenance, powered by telemetry data, allows for the identification of potential issues before they lead to failures, thereby reducing maintenance costs and improving operational efficiency.

This paper aims to explore the advancements in telemetry technology, its applications across various fields, and the future trends that could shape its development. By understanding the current state and potential of telemetry, we can better appreciate its critical role in modern technology and its impact on improving operational efficiencies and safety across different sectors.

%-----------------------------------------------------------------------------------------------

%-----------------------------------------------------------------------------------------------------------------------------------
\section{Statement of the Problem}

This study seeks to investigate some properties of decomposable hyper KS-semigroups in the context of strong, weak, quasi- and bi-hyper KS-ideals.

\section{Objectives of the Study}

In view of the above stated problem, we have the following objectives:
\begin{enumerate}
	\item To introduce the concept of strong, weak, quasi- and bi-hyper KS-ideals;
	\item To provide characterizations of strong, weak, quasi- and bi-hyper KS-ideals and investigate their relationships;
	\item To introduce the idea of decomposable hyper KS-semigroups and give some characterizations.
\end{enumerate}

\section{Significance of the Study}

The concept of hyperstructures is itself, a powerful mathematical tool since algebraic hyperstructures seem to occur very naturally in many areas of mathematics and even in other disciplines. 

\section{Scope and Limitations}

The primary motivation of this study lies within the structural properties of hyper 

\section{Definition of Terms}

\begin{description}

	\item[Data Logger] 
	An electronic device that records data over time or in relation to location either with a built-in instrument or sensor or via external instruments and sensors.
	
	\item[GPS Tracking] 
	Using the Global Positioning System to determine and track the precise location of a person, vehicle, or other asset.
	
	\item[Real-time Monitoring] 
	The process of continuously observing a system or process and immediately reporting any changes or anomalies.
	
	\item[Sensor] 
	A device that detects or measures a physical property and records, indicates, or otherwise responds to it.
	
	\item[Telemetry] 
	The process of recording and transmitting the readings of an instrument.
	
	\item[Wireless Communication] 
	The transfer of information between two or more points that are not connected by an electrical conductor.

\end{description}

}
