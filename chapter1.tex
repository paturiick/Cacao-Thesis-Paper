\chapter{INTRODUCTION}

{\baselineskip=2\baselineskip

	\section{Background of the Study}
\textit{Theobroma cacao}, widely known as cacao, is one of the most economically influential crops in the world. It serves as the primary raw material for the multibillion-dollar chocolate industry, supporting the livelihoods of approximately six million small-scale farmers globally. Once harvested, cacao seeds are processed to produce cocoa powder and cocoa butter—essential ingredients in a wide range of products, from confections and beverages to cosmetics and pharmaceuticals.

The global chocolate market is projected to grow significantly from 129.1 billion dollars in 2025, and up to 178.7 billion dollars by 2035 \citep{FutureMarketInsights2025ChocolateForecast}. However, \cite{Exquisito2025CocoaShortage} reported that West Africa, which supplies over 70\% of the world’s cocoa, have drastic declines in yield due to increasingly humid climatic conditions. 

In the Philippines, the Davao Region is the major producer of cacao which has been recognized as the “Chocolate Capital of the Philippines” \citep{PCAF2021}. Despite having favorable climate condition being a tropical region, it imports more than it produces which struggled to keep pace with the rising domestic demand. Nearby provinces such as Janog Cacao Plantation in Initao, Misamis Oriental, reports yield fluctuations between 100–500 kilograms of cacao beans, with 300 kilograms considered satisfactory and anything below 200 kilograms representing a deficit.

Notably, most yield losses occur during the rainy season, when excessive moisture promotes the spread of \textit{Phytophthora palmivora} (P.\textit{Palmivora}), the fungal pathogen responsible for black pod rot disease-locally known as \textit{butikol}. This disease has been documented to cause significant annual yield losses, posing a major challenge to cacao industry \citep{Avila2023}.

Currently, farmers rely primarily on visual inspection to detect early symptoms, such as small dark spots or discoloration, and remove affected pods immediately to prevent further contamination, since curative treatments are rarely used. However, due to the large size of farms and the labor-intensive nature of manual inspection, some infected pods go unnoticed.

The \cite{PhilCacaoRoadmap2021} reported that this method is labor intensive and error prone, often failing to detect infections in their early stages. These challenges were reaffirmed during a field interview with the owner of Janog Cacao Plantation in Initao, Misamis Oriental, a 5-hectare farm with over 2,500 cacao trees planted. According to the owner, routine inspection of cacao pods requires walking across the entire plantation, which takes up to three weeks to complete for one full cycle. This process becomes even more demanding during the rainy season. On average, only about 9 in 10 cacao trees are unhealthy, as the disease can quickly spread to nearby pods if infected ones are not pruned or removed promptly. 

To address these persistent threats, modern agriculture has increasingly turned to advanced technological solutions for early disease detection and intervention. Unmanned Aerial Vehicles (UAVs) and deep learning algorithms have emerged as powerful tools in precision agriculture, offering efficient and scalable monitoring of cacao plantations. UAVs, equipped with high-resolution cameras and multispectral sensors, can rapidly survey wide areas, while cutting-edge models such as You Only Look Once (YOLO) provide high-accuracy plant disease identification \citep{Vyas2023}. Early detection during the pre-harvest phase, as supported by \cite{Upadhyay2025,Yadav2024}, enables timely interventions to mitigate crop losses and ensure quality harvests.

Existing technological interventions for cacao disease detection, such as mobile applications that utilize image processing and machine learning, have made strides in bridging the gap. For instance, \cite{Tan2018} developed AuToDiDAC, an app designed to detect black pod rot, while \cite{Tovurawa2025} used convolutional neural networks (CNNs) to classify cacao leaf diseases. However, these solutions are predominantly dependent on static image inputs and close-range data collection. As noted by \cite{Taesiri2023}, such methods can cause models to focus only on the most discriminative regions of the plant, potentially missing early-stage infections or atypical symptoms that may be spread across the pod’s surface. Additionally, mobile-based approaches require farmers to manually photograph individual pods, which is laborious and impractical for cacao plantations, thereby limiting mobility and scalability.

With these challenges in mind, this study introduces a UAV-based cacao disease detection system that integrates the You Only Look Once (YOLO) object detection algorithm to address the limitations of static, close-range data collection. This approach enables a convenient monitoring of cacao plantations, such as the Janog Cacao Plantation, where manual inspection is physically demanding and time-consuming. Integrating deep learning model with UAV allows for the detection of disease symptoms across the entire pod surface that traditional methods may miss. Furthermore, the system incorporates geotagging through QGIS to efficiently map the locations of infected cacao pods, which are then visualized in a web-based platform that allows farmers to monitor the health of their plantations.

The proposed system will be developed followed by field tests at Janog Cacao Plantation in Initao, Misamis Oriental, to evaluate its performance and functionality in real-world agricultural environment.

\section{Statement of the Problem}

The Philippine cacao industry faces persistent challenges that hinder its ability to meet the demands of both the domestic and international market. Although the country has favorable climate conditions and fertile land, especially in the Davao Region, which represents 78\% of national production, it continues to fall short of its annual production target of 50,000 metric tons. According to the \cite{PhilCacaoRoadmap2021}, this shortfall is largely due to cacao diseases, particularly black pod disease caused by P.\textit{Palmivora}, which leads to post-harvest losses of up to 90\%.

Traditional detection methods rely on manual inspection, which is labor intensive, slow, and prone to human error, leading to delayed intervention and significant crop losses. Although existing studies explore machine learning and imaging technologies for cacao disease detection, they primarily use static imaging and mobile-based approaches, limiting monitoring and scalability.

\section{Objectives of the Study}

This study aims to design and develop a UAV-based system that predicts P.\textit{Palmivora} disease in cacao pods using deep learning model (object detection and disease detection) and GPS geotagging for precise location mapping. Specifically, it seeks to:

\begin{enumerate}
	\item Design and configure a UAV capable of autonomous navigation over cacao farms.
	\item Implement an object detection model for cacao pod detection and a classification model for identifying black pod rot disease, or P.\textit{Palmivora} infection.
	\item Develop and implement a monitoring system that tracks the UAV’s flight status and detection for cacao pod disease.
	\item Test the system’s detection accuracy, classification performance, geotagging precision, and overall operational efficiency.
\end{enumerate}

\section{Significance of the Study}
The study will be conducted at Janog Cacao Plantation in Initao, Misamis Oriental. This will hold significance for multiple sectors within the agricultural and technological landscape. By field-testing a UAV-based disease detection system in an actual cacao farm setting, the research may demonstrate the feasibility and practical benefits of integrating precision agriculture technologies into local farming operations.

For cacao farmers, particularly those managing plantations like Janog Cacao Plantation, the system may offer a convenient solution for early disease detection. It may enable timely intervention to prevent further contamination, helping reduce crop losses, improve yield quality and quantity.

For the cacao and chocolate industry, the study may contribute to sustaining both local and global markets by maintaining consistent raw material availability, controlling production costs, and supporting economic growth in cacao-dependent regions.

For the agricultural sector, the research may promote the modernization of farming through precision agriculture and remote sensing technologies. It may enhance productivity and sustainability, particularly in disease-prone areas where manual inspection is labor-intensive and inefficient.

For the government and policymakers, the study’s outcomes may serve as a basis for aligning technological innovation with the goals of the Philippine Cacao Industry Roadmap. It may provide valuable insights for policy formulation, funding assistance, and small to medium-scale implementation of technology-based interventions. Further, it may contributes to the realization of Sustainable Development Goals (SDG 8: Decent Work and Economic Growth, and SDG 15: Life on Land) by promoting sustainable agriculture, increasing farmer income, and fostering innovation.

Lastly, for future researchers, this study will serve as a reference for further exploration and enhancement of precision agriculture systems, UAV applications, and plant disease detection technologies. It will also provides a real-world benchmark for testing similar systems under tropical agricultural conditions.

\section{Scope and Limitations}
This study will focus on the development and testing of a UAV-based disease detection system for cacao farms in Initao, Misamis Oriental. The system integrates three major components: (1) a YOLO-based model for detecting cacao pods; (2) a YOLO-based model to detect visible symptoms of Black Pod Rot infection in cacao pods; (3) GPS and QGIS for efficient geolocation and mapping of affected trees; and (4) a web-based application for monitoring and visualization of detection results.

The study is limited to identifying external symptoms of black pod disease—such as visible pod rot and discoloration. Internal infections that do not manifest outwardly cannot be detected by the current implementation. Moreover, while the system can assist in identifying potentially infected pods, it does not automate subsequent farm management tasks such as pruning or removal of diseased pods, which must still be performed manually by farmers.

The imaging capability of the UAV is constrained by the use of a 720p camera, which may limit the level of visual detail captured and affect detection accuracy under certain environmental conditions. External factors such as lighting, weather conditions, and UAV flight stability may also influence detection performance.

Despite these constraints, the programmable nature of the UAV allows the use of its Software Development Kit (SDK) to define autonomous flight paths, enabling systematic data collection across cacao rows. However, the degree of automation depends on the reliability of the programmed commands and the environmental conditions encountered during flight missions.

These limitations define the operational boundaries of the proposed system and provide considerations for future improvements.

\section{Conceptual Framework}

\begin{figure}[H]
	\centering
  \caption{IPO Model}
	\label{fig:ipo}
	\includegraphics[width=1\textwidth]{figures/IPO.pdf}
\end{figure}

The study's conceptual framework is presented through an Input-Output-Process Model. The process begins with the input stage, where drones are used to capture images of cacao pods in the farm. These images serve as the primary source of information and are compared with sets of sample images showing both healthy and diseased pods. Along with the images, location data is recorded so that each detection can be linked to its exact place in the farm. 

The information collected then moves to the process stage. Here, the pod images are first checked to make them clear and consistent for analysis. After this, the system studies the images to identify whether pods are healthy or show signs of disease. The location data gathered during drone flights is combined with the results, allowing the system to create maps that show the areas where diseases may be present. This stage transforms raw images into useful information that can guide farmers in managing their crops.

In the output stage, the system produces results that are directly useful for farm management. These include the classification of cacao pods as healthy or infected, maps that point out exactly where the infected pods are located, and summary reports. The reports provide details such as the total number of pods detected, the percentage that are infected, and the specific times and places where infections were found.


\section{Definition of Terms}

For clarity and consistency, the following terms are defined as they are used in this study:

\begin{description}
	\item[Bounding Box] - A rectangular box generated by the YOLO model to localize and highlight detected objects, such as healthy or diseased cacao pods, within an image.
	
	\item[Convolutional Neural Network] - A deep learning architecture for visual analysis, serving as the backbone of the YOLO model in this study.
	
	
	\item[Dataset] - A structured collection of related data, such as images of cacao pods, used to train and evaluate deep learning models for disease detection in this study.

	\item[Deep Learning Algorithms] - A subset of machine learning algorithms, particularly neural networks, used to analyze large datasets and recognize patterns in images or other inputs, enhancing precision agriculture applications.

	\item[Disease Detection] - The process of identifying and diagnosing plant diseases, often involving technology such as image analysis and machine learning algorithms for early intervention.

	\item[Field Tests] - Practical trials conducted in real-world agricultural environments to assess the effectiveness and performance of the proposed UAV and deep learning-based system for detecting cacao pod diseases.

	\item[Geotagging] - The process of adding geographical location data, such as latitude and longitude, to images or data collected by UAVs, enabling spatial tracking and mapping of disease occurrences in cacao farms.
	
	\item[Ground Station] - The control and monitoring interface (web application) where UAV mission results, disease detections, and geotagged mappings are displayed for farmers.
	
	\item[Ground Sampling Distance] - The distance between pixel centers measured on the ground, used to estimate real-world size from UAV-captured images.

	\item[Image Processing] - The technique of manipulating and analyzing digital images using algorithms to extract meaningful information, often for detecting patterns such as plant diseases.

	\item[Object Detection] - A computer vision technique used to identify and locate multiple objects within an image, as performed by the YOLO model in this study.
	
	\item[\textit{Phytophthora palmivora} (P.\textit{Palmivora})] - A fungal pathogen responsible for causing black pod disease in cacao plants, which leads to significant yield losses in cacao production.

	\item[Pod] - Refers to the fruit of the cacao tree that contains cacao beans; it is the primary site for disease detection, particularly for symptoms caused by pathogens like P.\textit{Palmivora}.

	\item[Pre-harvest Detection] - The process of identifying signs of disease or stress in crops, specifically cacao pods, before they are harvested, allowing for timely intervention to prevent yield loss and improve crop quality.

	\item[QGIS] - An open-source Geographic Information System software that provides tools for geospatial data processing, mapping, and analysis. In this study, it is used for automating geotagging and visualizing infected cacao trees.

	\item[Static Imaging] - The process of capturing fixed, non-moving images, often used in traditional disease detection methods, which may miss early-stage infections or dispersed symptoms.
	
	\item[Telemetry] - The process of wirelessly transmitting data, such as flight status and image acquisition logs, from the UAV to the ground station.

	\item[Unmanned Aerial Vehicles (UAVs)] - Aerial devices, typically drones, that operate without a human pilot, often equipped with cameras and sensors, used for monitoring agricultural environments and gathering data for analysis.

	\item[You Only Look Once (YOLO)] - An advanced real-time object detection model that can quickly identify and classify objects within images, used for detecting diseases on plant surfaces in this study.
\end{description}
